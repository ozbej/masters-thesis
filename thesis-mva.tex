%----------------------------------------------------------------
%
%  File    :  thesis-embed.tex
%
%  Author  :  Keith Andrews, IICM, TU Graz, Austria
% 
%  Created :  22 Feb 96
% 
%  Changed :  19 Feb 2004
% 
%----------------------------------------------------------------

\chapter{Multidimensional Visual Analysis}

\label{chap:mva}

Multidimensional visual analysis (MVA) is a powerful tool for exploring and understanding complex datasets. By using visual representations of data, such as graphs, charts, and maps, analysts can quickly and easily identify trends, patterns, and relationships within the data. This can provide valuable insights and help decision-makers make informed choices based on the information at hand.

In recent years, the availability of large and complex datasets has increased dramatically, making it increasingly difficult for analysts to make sense of the data using traditional methods. MVA offers a solution to this problem by allowing analysts to quickly and easily explore the data and identify key trends and patterns.

Through the use of advanced visualization techniques, analysts can create interactive and engaging visualizations that help to bring the data to life. These visualizations can be easily shared with others, allowing for collaboration and discussion around the data.



\section{Multidimensional Visual Analysis Approaches}

MVA approaches are methods and techniques used to analyze and understand complex data sets using visual representations. These approaches typically involve the use of specialized software or tools that allow analysts to create and manipulate graphical representations of the data in order to uncover patterns, trends, and relationships. In this section, we will review some popular MVA approaches.


\subsection{Scatter Plots}

A scatter plot is a type of graph that is used to display the relationship between two numerical variables, to identify any potential trends or patterns in the data, and to identify outliers. It uses dots or markers to represent the values of the two variables, and position of each dot on the graph indicates the value of the two variables in a single observation.

To create a scatter plot, the values of one variable are plotted on the x-axis (horizontal axis) and the values of the other variable are plotted on the y-axis (vertical axis). The resulting graph will show a set of dots, with each dot representing a single observation. If there is a positive relationship between the two variables, the dots will tend to form a diagonal line that slopes upwards from left to right. If there is a negative relationship, the dots will tend to form a diagonal line that slopes downwards from left to right. If there is no relationship between the two variables, the dots will be scattered randomly across the graph.


\subsection{High Dimensional Projections}

Visualizing high-dimensional data can be challenging because it is difficult for the human brain to comprehend more than three dimensions. High dimensional projections are techniques that are used to reduce the number of dimensions in the data and represent it in a way that is easier to understand and interpret. High dimensional projections can be further split into \emph{linear} and \emph{non-linear} projections.


\subsubsection{Principal Component Analysis}

Principal Component Analysis (PCA) is a linear projection. PCA uses linear algebra to identify the underlying dimensions or factors in the data and project the data onto a lower-dimensional space \parencite{abdi2010principal}.


\subsubsection{Multi-Dimensional Scaling}

Multi-Dimensional Scaling (MDS) is a non-linear projection. MDS uses a distance metric to preserve the distances between data points in the high-dimensional space and project the data onto a lower-dimensional space \parencite{morrison2003fast}.


\subsubsection{t-Distributed Stochastic Neighbor Embedding}

t-distributed stochastic neighbor embedding (t-SNE) is a non-linear projection. t-SNE uses a probabilistic model to preserve the local structure of the data in the high-dimensional space and project the data onto a lower-dimensional space \parencite{van2008visualizing}.


\subsection{Parallel Coordinates}

Parallel coordinates are a type of chart that is used to visualize multi-dimensional data. In this type of chart, each data point is represented by a vertical line that extends across multiple axes. The axes are typically arranged in parallel. Each axis represents a different variable, and the position of a point on that axis indicates the value of that variable for the data point. This allows multiple variables to be compared and analyzed simultaneously \parencite{inselberg1990parallel}.

\subsection{Cluster Analysis}

Cluster analysis is a type of statistical technique used to identify groups of similar objects within a data set. It is a commonly used method for exploratory data analysis, and is often used as a way to gain insight into the underlying structure of the data. In cluster analysis, the data is divided into groups, or clusters, based on the similarity of the objects within each group. This allows analysts to identify common patterns and trends within the data, and to gain a better understanding of the relationships between the objects in the data set \parencite{duran2013cluster}.



\section{Multidimensional Visual Analysis Software}

MVA software is a type of computer program that is designed to help analysts visualize and analyze complex data sets. These programs typically include a wide range of tools and features that allow users to create and manipulate graphical representations of the data, such as scatter plots, parallel coordinates, and heat maps. By using these tools, analysts can quickly and easily explore the data and gain insights that might not be immediately apparent from looking at the raw data. MVA software is commonly used in fields such as business, finance, and marketing to help make data-driven decisions and uncover hidden trends in the data. In this section, we will review some popular MVA software programs.



